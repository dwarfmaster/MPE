\documentclass[12pt,a4paper] {article}

\usepackage[utf8]                               {inputenc}
\usepackage[T1]                                 {fontenc}
\renewcommand{\sfdefault}{phv}
\usepackage[francais]                           {babel}
\usepackage[a4paper, includefoot, margin=1.5cm] {geometry}
\usepackage                                     {titlesec}
\usepackage           {amsmath}
\usepackage           {amsfonts}
\usepackage           {amssymb}
\usepackage           {amsthm}
\usepackage           {mathrsfs}
\usepackage           {stmaryrd}
\usepackage           {fancybox}
\usepackage           {graphicx}
\usepackage           {float}
\usepackage           {pgf, tikz}
\usepackage{tabularx}
\usepackage{hyperref} % Formatage du pdf
\hypersetup{
    colorlinks    = false,
    breaklinks    = true,
    urlcolor      = blue,
    linkcolor     = red,
    bookmarksopen = true
}

\begin{document}
\title{Formulaire chimie}
\author{Luc Chabassier}
\maketitle

\section{Fonction d'état extensive}
\begin{tabularx}{\linewidth}{rlX}
    Énergie interne & $\Delta U = W + Q$ & \\
    Entalpie & $H = U + pV$ & \\
    Entropie & $\Delta S = S_{ech} + S_{créé}$ & \\
    Entalpie libre & $G = H - TS$ & aussi appelée potentiel thermodynamique, puisque minimal à l'équilibre. \\
\end{tabularx}

Pour une évolution monobare : \[\Delta H = Q_p\]

Pour un système qui n'est lieu ni de réaction chimique, ni de changement de phase : \[dG = Vdp - SdT\]

Pour une évolution monotherme et monobare entre deux états d'équilibre : \[\Delta G = -T_{ext}S_{créé}\]

\section{Grandeur de réaction}
$X \in \{H; S; G\}$, on a $X(T, p, \ldots, n_i, \ldots)$.

Donc $dX_{T,p} = \sum_i X_{m_i}dn_i$ avec $X_{m_i} = \frac{\partial X}{\partial n_i}$.

Si $A_i$ est pur, on pose $X_{m_i}^* = \frac{X}{n_i}$ la grandeur molaire. Dans le cas d'un mélange idéal, on a $X_{m_i}^* = X_{m_i}$.

Avancement d'une réaction : $d\xi = \frac{dn_i}{\nu_i}$, avec $\nu_i$ le coefficient stochiométrique algébrique. $\xi$ est une grandeur extensive en moles.

On pose $\Delta_r X = \frac{\partial X}{\partial \xi} = \sum_i \nu_i X_{m_i}$ car $X(T,p,\xi)$.

\subsection{Enthalpie de réaction}
On a $\Delta_r H = \frac{\partial H}{\partial \xi} = \sum_i \nu_i H_{m_i}$

Une réaction chimique étant monotherme et monobare, on a : $\Delta H = \xi \Delta_r H$.

On a donc si $\Delta_r H > 0$, la réaction est endothermique et si $\Delta_r H < 0$, la réaction est exothermique.

\section{Enthalpie standard de réaction}
État standard : état pour $p^0 = 1\text{ bar} = 10^5 Pa$. Pour un constituant gazeux, état sous $p^0$ à la même température du gaz parfait pur associé. Pour un constituant condensé, état de ce constituant pur dans le même état physique à la même température sous $p^0$.

État standard de référence : état standard du corps simple dans son état physique le plus stable dans les conditions standards.

Enthalpie standard de réaction : enthalpie de réaction du réactif (ou produit) dans son état standard, notée $H_{m_i}^0(T)$. On construit alors $\Delta_r H^0 = \sum_i \nu_i H_{m_i}^0$. On constate $\Delta_r H^0 \approx \Delta_r H$, d'où $\Delta H \approx \xi \Delta_r H^0$.

On a donc si $\Delta_r H^0 > 0$, la réaction est endothermique et si $\Delta_r H^0 < 0$, la réaction est exothermique.

Réaction de formation d'une espèce chimique : à une température fixée, réaction au cours de laquelle une mole de l'espèce dans son état standard est créée à partir des corps simples qui le constituent, chacun dans son état standard.

Enthalpie standard de formation d'un corps simple : enthalpie de réaction de la réaction de formation $\Delta_f H^0 = \Delta_r H^0$.

L'enthalpie standard de formation d'un corps simple dans son état standard de référence est nulle à toute température. Par convention, $\Delta_f H(H^+_{(aq)}) = 0$.

Loi de Hess : \[\Delta_r H^0 = \sum_i \nu_i \Delta_f H_i^0\]

On néglige la dépendance de $\Delta_r H^0$ en $T$.

\section{Entropie standard de réaction}
On étudie la variation d'entropie sur les différentes phases et les changements de phase :
\[ \Delta S_{m_i}^0 = C_{p(solide)}^0\ln\frac{T_{fus}}{T_0} + \frac{\Delta_r H_{fus}^0}{T_{fus}} + C_{p(liquide)}^0\ln\frac{T_{eb}}{T_{fus}} + \frac{\Delta_r H_{eb}^0}{T_{eb}} + C_{p(gaz)}^0\ln\frac{T_1}{T_{eb}} \]

On constate que $\Delta_r S^0$ a le même signe que $\Delta_r\nu_g$, il représente la variation de désordre.

Loi de Hess : \[\Delta_r S^0 = \sum_i \nu_i\Delta_f S^0_i\]

L'entropie standard de formation d'un corps simple dans son état standard de réaction est nulle à toute température. Par convention, $\Delta_f S^0(H^+_{(aq)}) = 0$.

\section{Entalpie libre standard de réaction}
Relation entre les $\Delta_r$ : $\Delta_r G^0 = \Delta_r H^0 - T\Delta_r S^0$.

Loi de Hess : $\Delta_r G^0 = \sum_i \nu_i\Delta_f G_i^0$.

\section{Potentiel chimique}
Potentiel chimique : $\mu_i = \frac{\partial G}{\partial n_i}$.

Expression de $G$ : $G = \sum_i \mu_i n_i$.

Activité : $\mu_i(T,p) = \mu_i^0(T) + RT\ln(a_i)$, où $a_i$ est l'activité du consituant $A_i$.

Pour un gaz parfait seul, $a_i = \frac{p}{p^0}$. Pour un mélange de gaz parfaits, $a_i = \frac{p_i}{p^0}$, où $p_i = x_ip$ est la pression partielle.

Pour une phase condensée pure $a = 1$.

Solution idéale : intéractions entre molécules du mélange de même intensité.

Dans une solution idéale, $a_i = x_i$ avec $x_i$ la fraction molaire de $A_i$.

Solution diluée : on distincte alors le solvant et le soluté.

Pour le solvant, $a_i = 1$. Pour un soluté, $a_i = \frac{c_i}{c^0}$, avec $c^0 = 1mol.L^{-1}$.

À l'équilibre physique, chaque corps pur a le même potentiel chimique dans les différentes phases.

Autre expression de l'enthalpie libre standard de réaction : $\Delta_r G^0 = \sum_i \nu_i \mu_i^0$.

\end{document}

