\documentclass[12pt,a4paper] {article}

\usepackage[utf8]                               {inputenc}
\usepackage[T1]                                 {fontenc}
\renewcommand{\sfdefault}{phv}
\usepackage[francais]                           {babel}
\usepackage[a4paper, includefoot, margin=1.5cm] {geometry}
\usepackage                                     {titlesec}
\usepackage           {amsmath}
\usepackage           {amsfonts}
\usepackage           {amssymb}
\usepackage           {amsthm}
\usepackage           {mathrsfs}
\usepackage           {stmaryrd}
\usepackage           {fancybox}
\usepackage           {graphicx}
\usepackage           {float}
\usepackage           {pgf, tikz}
\usepackage{tabularx}
\usepackage{hyperref} % Formatage du pdf
\hypersetup{
    colorlinks    = false,
    breaklinks    = true,
    urlcolor      = blue,
    linkcolor     = red,
    bookmarksopen = true
}

\begin{document}
\title{Formulaire chimie}
\author{Luc Chabassier}
\maketitle

\section{Définitions}
\begin{tabularx}{\linewidth}{|>{\bfseries}p{0.33\linewidth}|X|} \hline
    Énergie interne & $\Delta U = W + Q$ \\ \hline
    Enthalpie        & $H = U + pV$ \\ \hline
    Entropie        & $\Delta S = S_{ech} + S_{cree}$ \\ \hline
    Enthalpie libre  & $G = H - TS$ \\ \hline
    Potentiel thermodynamique & Autre nom de l'enthalpie libre, puisque minimale à l'équilibre. \\ \hline
    Grandeur de réaction & $X \in\{H; S; G\}, X(T, p, \ldots, n_i, \ldots)$ \\ \hline
    $X_{m_i}$ & $X_{m_i} = \frac{\partial X}{\partial n_i}$ \\ \hline
    $X_{m_i}^*$ & $X_{m_i}^* = \frac{X}{n_i}$ si $A_i$ est pur \\ \hline
    Coefficient stœchiométrique algébrique & $\nu_i$ \\ \hline
    Avancement d'une réaction & $d\xi = \frac{dn_i}{\nu_i}$, d'où $X(T,p,\xi)$ \\ \hline
    $\Delta_r X$ & $\Delta_r X = \frac{\partial X}{\partial \xi} = \sum_i \nu_i X_{m_i}$ \\ \hline
    Enthalpie de réaction & $\Delta_r H = \frac{\partial H}{\partial \xi} = \sum_i \nu_i H_{m_i}$ \\ \hline
    État standard & État pour $p^0$. Pour un constituant gazeux, état sous $p^0$ à la même température du gaz parfait pur associé. Pour un constituant condensé, état de ce constituant pur dans le même état physique à la même température sous $p^0$ \\ \hline
    État standard de référence & État standard du corps simple dans son état physique le plus stable dans les conditions standard. \\ \hline
    Enthalpie standard de référence & Enthalpie de réaction du réactif/produit dans son état standard, noté $H_{m_i}^0(T)$. \\ \hline
    $\Delta_r H^0$ & $\Delta_r H^0 = \sum_i \nu_i H_{m_i}^0$ \\ \hline
    Réaction de formation d'une espèce chimique & À température fixée, réaction au cours de laquelle une mole de l'espèce dans son état standard est créée à partir des crops simples qui le constituent, chacun dans son état standard \\ \hline
    Enthalpie standard de formation d'un corps simpe & Enthalpe de la réaction de formation $\Delta_f H^0 = \Delta_r H^0$ \\ \hline
    Potentiel chimique & $\mu_i = \frac{\partial G}{\partial n_i}$ \\ \hline
    Fraction molaire & $x_i$ \\ \hline
    Solution idéale & Intéraction entre molécules du mélange de même intensité \\ \hline
    Solution diluée & Solution avec un corps (le solvant) de fraction molaire fortement supérieure à celle des autres corps (solutés) \\ \hline
\end{tabularx}

Constantes : \\
\begin{center}\begin{tabular}{|l|l|} \hline
    \bf{Constante} & \bf{Valeur} \\ \hline
    $p^0$ & $1 bar = 10^5 Pa$ \\ \hline
    $c^0$ & $1mol.l^{-1}$ \\ \hline
\end{tabular}\end{center}

\section{Approximations}
\begin{itemize}
    \item On néglige la dépendance de $\Delta_r H^0$ en $T$.
\end{itemize}

\section{Relations}
\begin{tabularx}{\linewidth}{|l|X|} \hline
    \bf{Relation}    & \bf{Conditions} \\ \hline
    $\Delta H = Q_p$ & Monobare \\ \hline
    $dG = Vdp - SdT$ & Pas de réaction chimique, ni changement de base \\ \hline
    $\Delta G = -T_{ext}S_{cree}$ & Monotherme et monobare entre deux états d'équilibre \\ \hline
    $dX_{T,p} = \sum_i X_{m_i}dn_i$ & $\emptyset$ \\ \hline
    $X_{m_i} = X_{m_i}^*$ & Mélange idéal \\ \hline
    $\Delta H = \xi \Delta_r H$ & Monotherme et monobare \\ \hline
    $\Delta_r H^0 \approx \Delta_r H$ & $\emptyset$ \\ \hline
    $\Delta_f H^0(A) = 0$ & $A$ corps simple \\ \hline
    $\Delta_f H^0(H^+_{(aq)}) = 0$ & $\emptyset$ \\ \hline
    $\Delta_r H^0 = \sum_i \nu_i \Delta_f H_i^0$ & $\emptyset$ (loi de Hess) \\ \hline
    $\Delta_r S^0 = \sum_i \nu_i \Delta_f S_i^0$ & $\emptyset$ (loi de Hess) \\ \hline
    $\Delta_f S^0(A) = \Delta_f S^0(H^+_{(aq)}) = 0$ & $A$ corps simple \\ \hline
    $\Delta_r G^0 = \Delta_r H^0 - T\Delta_r S^0$ & $\emptyset$ \\ \hline
    $\Delta_r G^0 = \sum_i \nu_i\Delta_f G_i^0$ & $\emptyset$ (loi de Hess) \\ \hline
    $\mu_i(T,p) = \mu_i^0(T) + RT\ln(a_i)$ & $\emptyset$ \\ \hline
    $\forall (i,j), \mu_i = \mu_j$ & Équilibre physique dans la même phase \\ \hline
    $\Delta_r G^0 = \sum_i \nu_i\mu_i^0$ & $\emptyset$ \\ \hline
\end{tabularx}

Variation d'entropie sur les différentes phases et les changements de phase :
\[ \Delta S_{m_i}^0 = C_{p(solide)}^0\ln\frac{T_{fus}}{T_0} + \frac{\Delta_r H_{fus}^0}{T_{fus}} + C_{p(liquide)}^0\ln\frac{T_{eb}}{T_{fus}} + \frac{\Delta_r H_{eb}^0}{T_{eb}} + C_{p(gaz)}^0\ln\frac{T_1}{T_{eb}} \]

Activités : \\
\begin{center}\begin{tabular}{|l|l|} \hline
    \bf{Nature du corps} & \bf{activité $a_i$} \\ \hline
    Gaz parfait seul & $\frac{p}{p^0}$ \\ \hline
    Mélange de gaz parfaits & $\frac{p_i}{p^0}$ avec $p_i = x_ip$ \\ \hline
    Phase condensée pure & $1$ \\ \hline
    Solution idéale & $x_i$ \\ \hline
    Solvant en solution diluée & $1$ \\ \hline
    Soluté en solution idéale & $\frac{c_i}{c^0}$ \\ \hline
\end{tabular}\end{center}

\section{Interprétations}
\begin{tabularx}{\linewidth}{|l|X|} \hline
    \bf{Fait} & \bf{Interprétation} \\ \hline
    $\Delta_r H > 0$ & Réaction endothermique \\ \hline
    $\Delta_r H < 0$ & Réaction exothermique \\ \hline
    $\Delta_r H^0 > 0$ & Réaction endothermique \\ \hline
    $\Delta_r H^0 < 0$ & Réaction exothermique \\ \hline
    $\Delta_r S^0 > 0$ & Augmentation du désordre (ie $\Delta_r \nu_g > 0$) \\ \hline
    $\Delta_r S^0 < 0$ & Diminution du désordre (ie $\Delta_r \nu_g < 0$) \\ \hline
\end{tabularx}

\end{document}

