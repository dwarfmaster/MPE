\documentclass{article}

\usepackage[utf8]{inputenc}
\usepackage[T1]{fontenc}
\usepackage[french]{babel}
\usepackage[margin=2cm]{geometry}
\usepackage{amsmath}
\usepackage{amssymb}
\usepackage{mathrsfs}
\usepackage{mathtools}
\usepackage{stmaryrd}
\usepackage{listings}
\usepackage{hyperref}
\usepackage{xcolor}
\usepackage{tikz}

\newcommand{\N}{\mathbb{N}}
\newcommand{\M}{\mathscr{M}}
\newcommand{\T}{\tau}
\newcommand{\G}{\mathscr{G}}
\newcommand{\norm}[1]{\|#1\|}
\newcommand{\scal}{.}
\newcommand{\seg}[2]{\llbracket #1, #2 \rrbracket}
\DeclarePairedDelimiter{\ceil}{\lceil}{\rceil}
\DeclarePairedDelimiter{\floor}{\lfloor}{\rfloor}

\lstset{basicstyle=\normalsize,
    numbers=left,
    numberstyle=\color{black}\normalsize\textbf,
    numbersep=7pt,
    backgroundcolor=\color{white},
    commentstyle=\color{black}\textit,
    keywordstyle=\color{black}\textbf,
    rulecolor=\color{black},
    stringstyle=\color{black}\textit,
    identifierstyle=\color{black},
    frame=none,
    showstringspaces=true,
}

\hypersetup {
    colorlinks=true,
}

\renewcommand{\thesection}{Partie \Roman{section} -}

\title{Informatique\\-- Devoir Maison 1 --}
\author{Luc Chabassier}

\begin{document}
\maketitle
\section{Systèmes de pièces}
\begin{description}
    \item[I.A] Teste récursivement la validité du système :
        \begin{lstlisting}[language=Caml]
let rec est_un_systeme l = match l with
| []        -> false
| h::[]     -> h == 1
| h1::h2::t -> h1 > h2 && est_un_systeme (h2::t);;
        \end{lstlisting}
\end{description}

\section{Représentations de poids minimal}
\begin{description}
    \item[II.A] On procède par récurrence sur $x\in\N$ pour la deuxième inégalité.
        \begin{itemize}
            \item Si $x=0$, $k = 0_{\N^m}$ convient donc $\M(x) = 0$. Donc $\M(x) \leq x$.
            \item Soit $x\in\N$ tel que $\M(x)\leq x$. Soit $k\in\N^m$ tel que $k\scal c = x$ et $\norm{k} = \M(x)$. Posons $k' = k + e_m$. On a $k'\scal c = x+1$. Par conséquent $\M(x+1)\leq \norm{k'} = \M(x) + 1 \leq x+1$.
        \end{itemize}

        De plus, avec $\floor*{\frac{x}{c_1}}$ pièces, on peut atteindre un valeur maximum (en ne prenant que des pièces de valeurs $c_1$) de $n = \floor*{\frac{x}{c_1}} * c_1 \leq x$. Si l'inégalité est stricte, il faut donc au moins $\floor*{\frac{x}{c_1}} + 1 = \ceil*{\frac{x}{c_1}}$ pièces. Si il y a égalité, cela signifie que $\frac{x}{c_1} \in\N$, et donc $\ceil*{\frac{x}{c_1}} = \floor*{\frac{x}{c_1}}$. Donc $\forall k\in\N^m, k.c = x \implies \norm{k} \geq \ceil*{\frac{x}{c_1}}.$

        On a donc $\forall x \in\N^m, \ceil*{\frac{x}{c_1}} \leq \M(x) \leq x$.

    \item[II.B] Posons $c = (3,2,1)$. Dans ce système $4 = (0,2,0)\scal c = (1,0,1)\scal c$.

    \item[II.C] Soit $x\in\N^*$ et $j\in\seg{1}{m}$ tel que $c_j\leq x$. Soit $k\in\N^m$ un représentant minimal de $x-c_j$. On pose $k' = k + e_j$. On a $k'\scal c = x$ d'où $\M(x) \leq \norm{k'}$. De plus $\norm{k'} = \norm{k} + 1 = \M(x-c_j) + 1$.

        Donc $\M(x) \leq 1 + \M(x-c_j)$.

    \item[II.D] Soit $x\in\N^*$ et $j\in\seg{1}{m}$ tel que $c_j\leq x$.
        \paragraph{$\Leftarrow$} Supposons que $\M(x) = 1 + \M(x-c_j)$. Soit $k\in\N^m$ une représentation minimale de $x-c_j$. Posons $k' = k + e_j$. On a $k'\scal c = x$ et $\norm{k'} = \norm{k} + 1 = \M(x)$ donc $k'$ est une représentation minimal de $x$ de $j$-eme composante non nulle.
        \paragraph{$\Rightarrow$} Soit $k$ une représentation minimale de $x$ avec $k\scal e_j > 0$. Posons $k' = k-e_j$. $k'\scal c = x-c_j$ donc $\M(x-c_j) \leq \norm{k'} = \M(x) - 1$. Supposons que $\M(x-c_j) < \M(x) - 1$. Soit $k_2\in\N^m$ un représentant minimal de $x-c_j$. Posons $k_2' = k + e_j$.$k_2'$ est un représentant de $c$ donc $\M(x) \leq \norm{k_2'} = \norm{k_2} + 1 = \M(x-c_j) + 1 < \M(x)$ : contradiction. Donc $M(x) = 1 + M(x-c_j)$.

    \item[II.E] Soit $x\in\N, x > 1$. Soit $k\in\N^m$ un représentant minimal de $x$. $x\neq 0$ donc $\exists j\in\seg{1}{m} : k\scal e_j \neq 0 \Leftrightarrow \exists j\in\seg{1}{m} : \M(x) = 1 + \M(x-c_j)$. Couplé avec le résultat du \emph{II.C}, on a donc $\M(x) = 1 + \underset{j\in\seg{1}{m}, c_j\leq x}{\min} \M(x-c_j)$.

        Or, par définition de $s$ et par décroissance des $(c_j)$, on a $\{j\in\seg{1}{m} | c_j \leq x\} = \seg{s}{m}$.

        Donc $\M(x) = 1 + \underset{j\in\seg{s}{m}}{\min} \M(x-c_j)$

    \item[II.F] On crée un tableau de taille $x+1$, la première étant nulle et la deuxième valant 1. On parcourt pour $y$ allant de $2$ à $x$ et on remplit la $y$-eme case du tableau à l'aide de la formule précédente, qui requiert $2m$ comparaisons dans le pire des cas. On peut donc remplir le tableau en $O(mx)$.

    \item[II.G] On met en oeuvre l'algorithme décrit ci-dessus :
        \begin{lstlisting}[language=Caml]
let poids_minimaux x c =
    let cv = vect_of_list c in
    let m  = vect_length cv in
    let r  = make_vect (x+1) (-1) in
    r.(0) <- 0;
    for y = 1 to x do
        for i = 0 to m - 1 do
            if cv.(i) <= y then begin
                if r.(y) < 0 then
                    r.(y)  <- 1 + r.(y - cv.(i))
                else if r.(y) > 1 + r.(y - cv.(i)) then
                    r.(y)  <- 1 + r.(y - cv.(i))
            end
        done;
    done;
    tl (list_of_vect r);;
        \end{lstlisting}
\end{description}

\section{L'algorithme glouton}
\begin{description}
    \item[III.A] L'algorithme glouton :
        \begin{lstlisting}[language=Caml]
let rec glouton x c = match x with
| 0 -> []
| _ -> let h::t = c in
       let r = x / h in r :: glouton (x - r * h) t;;
        \end{lstlisting}

    \item[III.C] Si on rend moins de $\floor*{\frac{x}{c_1}}$ pièces de $c_1$, il faudra au moins $c_1$ pièces de $c_2 = 1$ pour compenser. Or $c_1 > 1$, donc il la représentation minimale est celle où l'on rend le plus possibles de pièces de $c_1$, soit $\floor*{\frac{x}{c_1}}$, ce qui correspond à la représentation gloutonne. Donc $(c_1, c_2)$ est canonique.

    \item[III.D] Soit $c = (10, 9, 1)$. On considère $x = 27$. En effet, $\G(27) = \norm{(2, 0, 7)} = 9$. Or $(0, 3, 0)\scal c = 27$ donc $\M(27) \leq 3$. Donc $c$ n'est pas canonique.

    \item[III.E] On procède par récurrence sur $n\in\N$.
        \begin{itemize}
            \item Si $n=0$, le résultat est immédiat.
            \item Soit $n\in\N$ tel que $c_n = (q^n, \ldots, q, 1)$ soit canonique. Considérons $c_{n+1} = (q^{n+1}, q^n, \ldots, q, 1)$.

                Soit $x\in\N$. Si $x<q^{n+1}$, le résultat est immédiat en utilisant la propriété de récurrence. Soit $k$ une représentation minimale de $x$ dans $c_{n+1}$. On a $\M_{c_{n+1}}(x) = k\scal e_1 + \M_{c_n}(x - (k\scal e_1)q^{n+1})$ car si la représentation de $x' = x - (k\scal e_1)q^{n+1}$ n'est pas optimale dans $c_n$, on peut réduire la représentation de $x$ dans $c_{n+1}$. Par propriété de récurrence, $\M_{c_n}(x') = \G_{c_n}(x')$. Posons $y = x - \floor*{\frac{x}{q^{n+1}}}$ et $d =\floor*{\frac{x}{q^{n+1}}} - k\scal e_1$. Supposons que $d > 0$. On a $x' = u + d.q^n$ donc $\T_{c_n}(x') = \T_{c_n}(y) + dq.e_1$. Donc $\G_{c_n}(x') = \G_{c_n}(y) + dq$. On a alors $\M_{c_{n+1}}(x) = k\scal e_1 + \G_{c_n}(y) + dq$. Or on peut réduire $d$ de $1$ en rendant un pièce de $q^{n+1}$ plutôt que $q$ pièces de $q^n$. En itérant le procédé, on trouve que la valeur optimale pour $k\scal e_1$ est $\floor*{\frac{x}{q^{n+1}}}$.
        \end{itemize}

    \item[III.F] On commence par montrer un certain nombre d'inégalités pour $k=(k_1,\ldots,k_8)$ un représentation minimale de $x\in\N$ :\begin{itemize}
            \item Si $k_8\geq 2$, on peut réduire le nombre de pièces en rendant une pièce de $2$ plutôt que $2$ pièces de $1$. Donc $k_8\leq 1$.
            \item Suivant le même raisonnement, on montre que $k_7\leq 4$. On traite plusieurs cas maintenant :\begin{itemize}
                    \item Si $(k_7,k_8) = (4,0)$, on peut représenter la même quantité avec moins de pièces avec $(k_6, k_7, k_8) = (1, 1, 1)$, donc cette représentation n'est pas optimale.
                    \item Si $(k_7,k_8) = (3,0)$, on peut réduire de même avec $(k_6,k_7,k_8) = (1, 0, 1)$.
                    \item Si $(k_7,k_8) = (4,1)$, on peut réduire avec $(k_6,k_7,k_8) = (1,2,0)$.
                    \item Si $(k_7,k_8) = (3,1)$, on peut réduire avec $(k_6,k_7,k_8) = (1,1,0)$.
                \end{itemize}
                On peut donc déduire l'inégalité $2k_7 + k_8 \leq 4$.
            \item On peut de même montrer que $k_6\leq 1$, donc $5k_6 + 2k_7 + k_8 \leq 9$.
            \item On montre de même que $k_5\leq 1$ d'où $10k_5 + 5k_6 + 2k_7 + k_8 \leq 19$.
            \item On montre aussi que $k_4 \leq 4$. On traite les même cas que pour $k_7$ mais avec les dizaines et on trouve $20k_4 + 10k_5 + 5k_6 + 2k_7 + k_8 \leq 49$.
            \item On montre que $k_3 \leq 1$, soit $50k_3 + 20k_4 + 10k_5 + 5k_6 + 2k_7 + k_8 \leq 99$.
            \item On montre enfin que $k_2 \leq 1$, soit $100k_2 + 50k_3 + 20k_4 + 10k_5 + 5k_6 + 2k_7 + k_8 \leq 199$.
        \end{itemize}

        Or si deux représentations de $x$ vérifient les inégalités ci-dessus, elles sont égales d'après l'unicité de la division euclidienne. On constate que la représentation gloutonne vérifie les inégalités ci-dessus, donc elle coïncide avec un représentation minimale. C'est donc une représentation minimale, donc le système euros est canonique.

    \item[III.G] On a $\G(48) = \norm{(1, 0, 1, 1, 0, 0)} = 3$, or $(0, 2, 0, 0, 0, 0)\scal c = 48$ donc $\M(48) \leq 2 < \G(48)$. Le système du Royaume-Uni d'avant la réforme de 1971 n'est pas canonique.
\end{description}

\section{L'algorithme de Kozen et Zaks}
\begin{description}
    \item[IV.A] Supposons $x\leq c_{m-2} + 1$.\begin{itemize}
            \item Si $x = c_{m-3} = c_{m-2} + 1$, alors $k = (0,\ldots,1, 0, 0)$, la représentation gloutonne, est minimale.
            \item Sinon, $x$ s'exprime dans $(c_{m-1}, c_m)$, qui est canonique, donc la représentation gloutonne est minimale.
        \end{itemize}
        % TODO : deuxième inégalité

    \item[IV.B] Posons $x = 2q$. On a $\T(x) = (1, 0, q-1) \implies \G(x) = q \geq 3$. Or $(0, 2, 0)$ convient, donc $\M(x) \leq 2 < \G(x)$, donc $x$ est un contre exemple : $c$ n'est pas canonique.

        Soit $x'\in\seg{q+2}{2q-1}$. On a $\G(x') = \norm{(1,0,x' - (q+1))} = x' - q$. Soit $k$ un représentation minimale de $x'$. Si $k\scal e_1 = 1$, alors c'est la représentation gloutonne. Supposons maintenant que $k=(0,a,b)$. On a $x' = qa + b$. Or $(q,1)$ est canonique donc $a = \floor*{\frac{x}{q}} = 1$ et $b = x' - q$. Donc $\norm{k} = \G(x')$. On en déduit que $\M(x') = \G(x')$, soit $x'$ n'est pas un contre exemple.

        $x = 2q$ est donc le plus petit contre-exemple.

    \item[IV.C] Posons $\alpha(q) = 2q - 2$. Posons $x = \alpha(q) + 2 = 2q$. $(0,2,0)$ est une représentation de $x$ donc $\M(x) \leq 2$. Or $\G(x) = \norm{(1,0,2)} = 3$. Donc $x$ est un contre exemple. Or $x = c_{m-2} + 2$ donc $x$ est le plus petit contre exemple.

    \item[IV.D] On vient de montrer que les inégalités du \emph{II.A} sont optimales.

    \item[IV.E] Soit $x\in\N$ un témoin et $i\in\seg{1}{m}$ tel que $c_i < x$ et $\G(x-c_i) < \G(x) - 1$. Posons $k = \T(x-c_i) + e_i$. $\norm{k} = \G(x-c_i) + 1 < \G(x)$ et $k\scal c = x$ donc $\M(x) \leq \norm{k} < \G(x)$, donc $x$ est un contre-exemple.

    \item[IV.F] On se place dans le système $c = (5,4,1)$ et on considère $n=12$. On a $(0,3,0)\scal c = 12$ donc $\M(12) \leq 3$. De plus $\G(12) = 4$. Donc $12$ est un contre exemple. Calculons maintenans :\begin{itemize}
            \item $\G(12-5) = \G(7) = 3  \geq \G(12) - 1$
            \item $\G(12-4) = \G(8) = 4  \geq \G(12) - 1$
            \item $\G(12-1) = \G(11) = 3 \geq \G(12) - 1$
        \end{itemize}

        Donc $12$ n'est pas un témoin : il existes des contre-exemples qui ne sont pas de témoins.

    \item[IV.G] Soit $x\in\N$ le plus petit contre-exemple de $c$. Supposons que $\forall i\in\seg{1}{m}, c_i < x \implies \G(x-c_i) \geq \G(x) - 1$, avec $\G(x-c_i) = \M(x-c_i)$ car $x$ est le plus petit contre-exemple. Donc, en reprenant la notation de $s$ :
        \[ \forall i\in\seg{s}{m}, \G(x) \leq 1 + \M(x + c_i)
            \implies \G(x) \leq 1 + \underset{i\in\seg{1}{m}}{\min} \M(x-c_i) = \M(x) \]

        Contradiction avec $x$ contre-exemple. Donc $x$ est un témoin.

    \item[IV.H] Appliquons l'algorithme :\begin{itemize}
            \item $c = (q,2,1), q \geq 3$. On a $c_{m-2} + 2 = q + 2 > c_1 + c_2 - 1 = q + 1$, donc il ne peut y avoir de contre-exemple, le système est canonique.
            \item $c = (7,4,1)$. On a $c_{m-2} + 2 = 9$ et $c_1 + c_2 - 1 = 10$. On a :\begin{itemize}
                    \item $\G(9-7) = 2 \geq \G(9) - 1$, $\G(9-4) = 2 \geq \G(9) - 1$ et $\G(9-1) = 3 \geq \G(9) - 1$
                    \item $\G(10-7) = 3 \geq \G(10) - 1$, $\G(10-4) = 3\geq \G(10) - 1$ et $\G(10-1) = 3 \geq \G(10) - 1$
                \end{itemize}
                Il n'y a pas de contre-exemple, donc le système est canonique.
            \item $c = (6, 5, 1)$. On a $c_{m-2} + 2 = 8$ et $c_1 + c_2 - 1 = 10$. On a :
                \begin{itemize}
                    \item $\G(8 - 6) \geq \G(8) - 1$, $\G(8 - 5) \geq \G(8) - 1$ et $\G(8 - 1) \geq \G(8) - 1$
                    \item $\G(9 - 6) \geq \G(9) - 1$, $\G(9 - 5) \geq \G(9) - 1$ et $\G(9 - 1) \geq \G(9) - 1$
                    \item $\G(10 - 6) = 4 \geq \G(10) - 1 = 4$, mais $\G(10 - 5) = 1 < \G(10) - 1$.
                \end{itemize}
                On a fait apparaitre un témoin donc un contre-exemple, le système n'est pas canonique.
        \end{itemize}

    \item[IV.I] On utilise l'algorithme \texttt{glouton} précédent pour calculer la décomposition gloutonne. On code aussi la fonction \texttt{lsum} qui prend une liste d'entier et renvoie la somme de ces entiers :
        \begin{lstlisting}[language=Caml]
let rec lsum l = match l with
| []   -> 0
| h::t -> h + lsum t;;
        \end{lstlisting}

        On peut maintenant coder l'algorithme de Kozen et Zaks :
        \begin{lstlisting}[language=Caml]
let kozen_zaks c =
    let v = vect_of_list c in
    let m = vect_length v in
    let d = v.(m-3) + 2 and f = v.(0) + v.(1) + 1 in
    let r = ref true in
    for x = d to f do
        for i = 0 to m - 1 do
            if v.(i) < x then begin
                let g = glouton x c and g2 = glouton (x - v.(i)) c in
                if lsum g < lsum g2 - 1 then r := false;
            end;
        done;
    done;
    !r;;
        \end{lstlisting}

    \item[IV.K] L'algorithme \texttt{glouton} ayant une complexité en $O(m)$ et la boucle intérieure bouclant $m$ fois, la boucle intérieure a une complexité en $O(m^2)$. Il s'agit maintenant d'exiber des systèmes tels que $c_1 + c_2 - c_{m-2} - 3$ grandisse exponentiellement avec $m$.

        Considérons les systèmes $c_m = (q^m, \ldots, q, 1)$. Ces systèmes sont canoniques (déjà montré). Or $c_1 + c_2 - c_{m-2} - 3 = q^m\left(1 - \frac{1}{q^2} - \frac{1}{q^{m-2}} - \frac{3}{q^m}\right) \underset{m\rightarrow +\infty}{\sim} q^m\left(1 - \frac{1}{q^2}\right)$. Donc sur ces systèmes la complexité est en $O(m^2q^m)$, soit sur-exponentielle.

        Considérons les systèmes $c_m = (q^m, q^{m-1}, \ldots q^3, q^2+1, q^2, 1)$. Ces systèmes ne sont pas canoniques, car sinon $(q^2+1,q^2,1)$ le serait. Or $c_1 + c_2 - c_{m-2} - 3 = q^m\left(1 + \frac{1}{q} - \frac{1}{q^{m-2}} - \frac{4}{q^m}\right) \underset{m\rightarrow +\infty}{\sim} \left(1 - \frac{1}{q}\right)q^m$. Donc la complexité est là aussi en $O(m^2q^m)$.
\end{description}

\section{Répartition de paire de skis}
\begin{center}
    \begin{tikzpicture}
        % Axis j
        \draw (0,0) -- (2,0);
        \draw (2,0) [dotted,thick] -- (5,0);
        \draw (5,0) [arrows=->] -- (8,0) node [right] {$j$};
        % Axis i
        \draw (0,0) -- (0,2);
        \draw (0,2) [dotted,thick] -- (0,5);
        \draw (0,5) [arrows=->] -- (0,7) node [above] {$i$};

        % Beginning point
        \draw (0.1,6) -- (-0.1,6) node [left] {$n-1$};
        \draw (7,0.1) -- (7,-0.1) node [below] {$m-1$};
        \fill [red] (7,6) circle[radius=2pt];
        \draw [red] (0,6) [loosely dotted,thick] -- (7,6);
        \draw [red] (7,0) [loosely dotted,thick] -- (7,6);

        % Dependencies
        \draw [arrows=->,blue,thick] (6.8,6) -- (6.2,6);
        \draw [arrows=->,blue,thick] (6.8,5.8) -- (6.2,5.2);
        \fill [blue] (6,6) circle[radius=2pt];
        \fill [blue] (6,5) circle[radius=2pt];
        \draw [arrows=->,blue,thick] (5.8,5.8) -- (5.2,5.2);
        \draw [arrows=->,blue,thick] (5.8,5) -- (5.2,5);
        \fill [blue] (5,5) circle[radius=2pt];

        \draw [blue] (4.8,4.8) [dotted,thick] -- (2.2,2.2);
        \draw [blue] (5.8,4.8) [dotted,thick] -- (2.2,1.2);
        \fill [blue] (2,2) circle[radius=2pt];
        \draw [arrows=->,blue,thick] (1.8,1.8) -- (1.2,1.2);
        \fill [blue] (2,1) circle[radius=2pt];
        \draw [arrows=->,blue,thick] (1.8,0.8) -- (1.2,0.2);
        \draw [arrows=->,blue,thick] (1.8,1) -- (1.2,1);
        \fill [blue] (1,1) circle[radius=2pt];
        \draw [arrows=->,blue,thick] (0.8,0.8) -- (0.2,0.2);
        \fill [blue] (1,0) circle[radius=2pt];
        \draw [arrows=->,blue,thick] (0.8,0) -- (0.2,0);
        \fill [blue] (0,0) circle[radius=2pt];
    \end{tikzpicture}
\end{center}

Le points rouge est celui dont la valeur nous intéresse. Les flèches bleus indiquent les dépendances. On ne doit donc calculer que les valeurs des points colorés en bleu.
Il y en a $n(m-n+1)$ : on doit calculer $n$ lignes, et sur chaque seuls ceux entre la diagonale passant par l'origine et celle passant par $(m,n)$ nous intéressent.

Plutôt que de créer un tableau avec $nm$ cases et les remplir à une valeur indiquant l'absence de calcul, on peut se contenter d'utiliser un tableau de taille $m-n+1$, représentant une ligne. On calcule ensuite la ligne suivante dans le même tableau selon les $j$ croissants, ce qui est possible car pour calculer un points, on n'utilise que la valeur sur $j-1$ sur la même ligne (calculé précédemment) ou la valeur sur $(j-1,i-1)$, qui n'est pas encore remplacée par la nouvelle valeur.

\begin{lstlisting}[language=Caml]
let paires_ski h s =
    let m = vect_length s in
    let n = vect_length h in
    let l = make_vect (m-n+1) (abs (h.(0) - s.(0))) in
    let d = make_matrix n (m-n+1) false in
    for j = 1 to m - n do
        if l.(j-1) < abs (h.(0) - s.(j)) then begin
            d.(0).(j) <- true;
            l.(j) <- l.(j-1);
        end else l.(j) <- abs (h.(0) - s.(j))
    done;
    for i = 1 to n - 1 do
        l.(0) <- l.(0) + abs (h.(i) - s.(i));
        for j = 1 to m - n do
            let t = l.(j) + abs (h.(i) - s.(i+j)) in
            if t < l.(j-1) then l.(j) <- t
            else begin
                l.(j) <- l.(j-1);
                d.(i).(j) <- true;
            end;
        done;
    done;
    let r = make_vect n 0 in
    let rec attribution i j = match (i,j) with
    | (0,0)         -> r.(0) <- 0
    | (0,_)         -> if d.(0).(j) then attribution i (j-1) else r.(0) <- j
    | _ when j == 0 -> begin
                r.(i) <- i;
                attribution (i-1) j;
            end;
    | _             -> if d.(i).(j) then attribution i (j-1)
                       else begin
                           r.(i) <- i + j;
                           attribution (i-1) j;
                       end;
    in attribution (n-1) (m-n); r;;
\end{lstlisting}

\end{document}

